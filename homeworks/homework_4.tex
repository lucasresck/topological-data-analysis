\documentclass{article}
\usepackage[utf8]{inputenc}

\title{Homework 4\\
    \large{Topological Data Analysis with Persistent Homology}}
\author{Lucas Emanuel Resck Domingues\\    
    Professor: Raphaël Tinarrage\\\\
    {Escola de Matemática Aplicada}\\
    {Fundação Getulio Vargas}}
\date{\today}

% \usepackage{natbib}
\usepackage{graphicx}
\usepackage{amsmath}
\usepackage{amsfonts}

\begin{document}

    \maketitle

    \noindent \textbf{Exercise 20.} \textit{What is the Euler characteristic of a sphere of dimension 1? 2? 3?}

    \begin{itemize}
        \item A sphere of dimension 1 is a circunference, so its triangulation can be writen with 3 vertices and simplicial complex $$K = \{[0], [1], [2], [0, 1], [0, 2], [1, 2]\}$$
            The Euler characteristic is $3 - 3 = 0$.

        \item A sphere of dimension 2 is a conventional sphere in $\mathbb{R}^3$. We will create a simplicial complex that looks like a tetrahedron, and its topological realization will be homeomorphic to the sphere. $V = \{0, 1, 2, 3\}$,
            \begin{gather*}
                K = \{[0, 1, 2], [0, 1, 3], [0, 2, 3], [1, 2, 3], [0, 1], [0, 2], \\ [0, 3], [1, 2], [1, 3], [2, 3], [0], [1], [2], [3]\}
            \end{gather*}
            The Euler characteristic is $4 - 6 + 4 = 2$.

        \item A sphere of dimension 3 is more difficult to assimilate.
            We will add one vertex and write down all possibilities of simplices of dimension 3:
            $V = \{0, 1, 2, 3, 4\}$, $$[1, 2, 3, 4], [0, 2, 3, 4], [0, 1, 3, 4], [0, 1, 2, 4], [0, 1, 2, 3]$$
            If we construct the resulting simplicial complex of it (consisting of subsets), we will have
            \begin{itemize}
                \item 5 simplices with dimension 3
                \item $\dfrac{5 \cdot 4 \cdot 3}{3!}$ simplices with dimension 2 (combinations)
                \item $\dfrac{5 \cdot 4}{2!}$ simplices with dimension 1
                \item 5 simplices with dimension 0
            \end{itemize}
            Therefore, the Euler characteristic is $5 - 10 + 10 - 5 = 0$.\\

    \end{itemize}

    \noindent \textbf{Exercise 21.} \textit{Using the previous exercise, show that $\mathbb{R}^3$ and $\mathbb{R}^4$ are not homeomorphic.} \\

    Suppose, by contradiction, that $\mathbb{R}^3 \simeq \mathbb{R}^4$, that is, there exists a homeomorphism $f$ between these two sets.
    So $$g \colon \mathbb{R}^3 \backslash \{0\} \longrightarrow \mathbb{R}^4 \backslash \{f(0)\}, \ x \longmapsto f(x)$$
    is a homeomorphism too.
    We can deduce there exist a homeomorphism $$h \colon \mathbb{R}^3 \backslash \{0\} \longrightarrow \mathbb{R}^4 \backslash \{0\}, \ x \longmapsto f(x) - f(0)$$
    so we are good to go.

    We already know that the unity sphere of dimension $n$ is homotopic equivalent to $\mathbb{R}^{n+1} \backslash \{0\}$.
    Hence, $\mathbb{S}_2 \approx \mathbb{R}^3 \backslash \{0\}$ and $\mathbb{S}_3 \approx \mathbb{R}^4 \backslash \{0\}$.
    Because $\mathbb{R}^3 \backslash \{0\} \simeq \mathbb{R}^4 \backslash \{0\}$ (by initial assumption), we have $\mathbb{R}^3 \backslash \{0\} \approx \mathbb{R}^4 \backslash \{0\}$ and, therefore, $\mathbb{S}_2 \approx \mathbb{S}_3$.

    Two homotopic spaces must have the same Euler characteristic, but we already know $\chi(\mathbb{S}_2) = 2$ and $\chi(\mathbb{S}_3) = 0$.
    This contradiction leads us to the fact that $\mathbb{R}^3 \not\simeq \mathbb{R}^4$.

\end{document}
