\documentclass{article}
\usepackage[utf8]{inputenc}

\title{Homework 5\\
    \large{Topological Data Analysis with Persistent Homology}}
\author{Lucas Emanuel Resck Domingues\\    
    Professor: Raphaël Tinarrage\\\\
    {Escola de Matemática Aplicada}\\
    {Fundação Getulio Vargas}}
\date{\today}

% \usepackage{natbib}
\usepackage{graphicx}
\usepackage{amsmath}
\usepackage{amsfonts}

\begin{document}

    \maketitle

    \noindent \textbf{Exercise 28.} \textit{Let $(G, +)$ be a group, potentially non-commutative.
    Prove that
    $$\forall g \in G,\ g + g = 0 \Longrightarrow \textrm{$G$ is commutative.}$$}

    Take two arbitrary point $a, b \in G$.
    Follow this:
    \begin{align*}
        (a + b) + (a + b) &= 0 \\
        a + b + a + b &= 0 \ \ \ \ \textrm{(associativity)}\\
        a + b + a + b + b &= b \\
        a + b + a + 0 &= b \ \ \ \ \textrm{(our assumption)}\\
        a + b + a &= b \ \ \\
        a + b + a + a &= b + a \\
        a + b + 0 &= b + a \\
        a + b &= b + a
    \end{align*}

    We got where we wanted to go.    

\end{document}
