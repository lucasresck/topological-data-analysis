\documentclass{article}
\usepackage[utf8]{inputenc}

\title{Homework 2\\
    \large{Topological Data Analysis with Persistent Homology}}
\author{Lucas Emanuel Resck Domingues\\    
    Professor: Raphaël Tinarrage\\\\
    {Escola de Matemática Aplicada}\\
    {Fundação Getulio Vargas}}
\date{\today}

% \usepackage{natbib}
\usepackage{graphicx}
\usepackage{amsmath}
\usepackage{amsfonts}

\begin{document}

    \maketitle

    \noindent\textbf{Exercise 8.} \textit{Show that the topological spaces $\mathbb{R}^n$ and $\mathcal{B}(0, 1) \subset \mathbb{R}^n$ are homeomorphic.}\\

    Let $f$ be the map
    \begin{align*}
        f: \ &\mathcal{B}(0, 1) \to \mathbb{R}^n\\
        &x \mapsto \begin{cases}
            \dfrac{x}{\lVert x \lVert} \tan\left(\dfrac{\pi}{2}\lVert x \lVert \right), &x \ne 0\\
            0, &x = 0
        \end{cases}
    \end{align*}

    If we take two different $x_1 \ne x_2$:
    \begin{itemize}
        \item if one is multiple of the other: they will have different modules, because $\tan\left(\dfrac{\pi}{2}\lVert x_1 \lVert \right) \ne \tan\left(\dfrac{\pi}{2}\lVert x_2 \lVert \right)$
        \item if one is not multiple of the other: they will continue not being multiple of the other.
    \end{itemize}
    So the function is injective.

    For each point $y \ne 0 \in \mathbb{R}^n$, we can reach it by taking $x = \dfrac{y}{\lVert y \lVert} \dfrac{2}{\pi}\arctan\lVert y \lVert$.
    So the function is surjective. We conclude $f$ is bijective.

    Considering the above, we arrive with the following inverse function:
    \begin{align*}
        f^{-1}: \ &\mathbb{R}^n \to \mathcal{B}(0, 1)\\
        &y \mapsto \begin{cases}
            \dfrac{y}{\lVert y \lVert} \dfrac{2}{\pi}\arctan\lVert y \lVert, &y \ne 0\\
            0, &y = 0
        \end{cases}
    \end{align*}

    The two functions are clearly continuous out of zero. In order to check if they are continuous at zero, we will consider the module of the functions.
    
    \begin{align*}
            \lVert f(x) \lVert &= \begin{cases}
            \left\lVert \dfrac{x}{\lVert x \lVert} \tan\left(\dfrac{\pi}{2}\lVert x \lVert \right) \right\lVert, &x \ne 0\\
            0, &x = 0
        \end{cases}\\
        &= \begin{cases}
            \tan\left(\dfrac{\pi}{2}\lVert x \lVert \right), &x \ne 0\\
            0, &x = 0
        \end{cases}\\
        &\to_{x \to 0} 0
    \end{align*}

    The same for $f^{-1}$. So the functions are continuous at zero, therefore continuous. This way, $f$ is a homeomorphism.\\

    \noindent\textbf{Exercise 11.} \textit{Show that $[0, 1)$ and $(0, 1)$ are not homeomorphic.}\\

    Suppose there is a homeomorphism $f$ between $[0, 1)$ and $(0, 1)$.
    So $$g: [0, 1)\backslash\{0\} = (0, 1) \to (0, 1)\backslash\{f(0)\}$$
    must be a homeomorphism too. $(0, 1)$ has one connected component, and $(0, 1)\backslash\{f(0)\}$, because of $f(0) \in (0, 1)$, has two connected components.
    This is a contradiction.

\end{document}
