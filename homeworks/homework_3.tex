\documentclass{article}
\usepackage[utf8]{inputenc}

\title{Homework 3\\
    \large{Topological Data Analysis with Persistent Homology}}
\author{Lucas Emanuel Resck Domingues\\    
    Professor: Raphaël Tinarrage\\\\
    {Escola de Matemática Aplicada}\\
    {Fundação Getulio Vargas}}
\date{\today}

% \usepackage{natbib}
\usepackage{graphicx}
\usepackage{amsmath}
\usepackage{amsfonts}

\begin{document}

    \maketitle

    \noindent\textbf{Proposition 3.8.} \textit{Let $f \colon    \mathbb{R}^n \to X$ be a continuous map.
    Then $f$ is homotopic to a constant map.} \\

    \noindent\textbf{Exercise 12.} \textit{Prove the previous proposition.} \\

    Consider the continuous application:
    \begin{align*}
        F \colon \mathbb{R}^n \times [0, 1] &\longrightarrow X \\
        (x, t) &\longmapsto f(xt)
    \end{align*}
    We have $F(\cdot, 0) \colon x \to f(0)$ constant and $F(\cdot, 1) = f$. \\ \\

    \noindent\textbf{Exercise 13.} \textit{Let $f \colon \mathbb{S}_1 \to \mathbb{S}_2$ be a continuous map which is not surjective.
    Prove that it is homotopic to a constant map.} \\

    Take $x_0$ outside of the image of $f$ but on $\mathbb{S}_2$, that is, $x_0 \in \mathbb{S}_2$ such as $x_0 \notin f(\mathbb{S}_1)$.
    Our constant map will be $g \colon x \mapsto -x_0$.

    Consider the following continuous application:
    \begin{align*}
        F \colon \mathbb{S}_1 \times [0, 1] &\longrightarrow \mathbb{S}_2 \\
        (x, t) &\longmapsto \dfrac{(1-t) f(x) - t x_0}{\lVert (1-t) f(x) - t x_0 \lVert}
    \end{align*}

    We are doing a convex combination between the points of the image of $f$ and our point $-x_0$. Because $x_0$ is not on the image, the convex combination never passes through the origin, hence the denominator is never zero, therefore $F$ is continuous.

    Note that $F(\cdot, 0) = f$ and $F(\cdot, 1) \colon x \mapsto -x_0$. \\\\

    \noindent\textbf{Exercise 14.} \textit{Show that being homotopic is a transitive relation between maps:
    for every triplet of maps $f, g, h \colon X \to Y$, if $f, g$ are homotopic and $g, h$ are homotopic, than $f, h$ are homotopic.} \\

    Because $f, g$ are homotopic, we know there must exist a homotopy.
    Let's call it $F \colon X \times [0, 1] \to Y$.
    The same for $g, h$:
    $G \colon X \times [0, 1] \to Y$.
    Consider now the following continuous application:
    \begin{align*}
        H \colon X \times [0, 1] &\longrightarrow Y \\
        (x, t) &\mapsto (1 - t) F(x, 0) + t G(x, 1)
    \end{align*}

    Because $F, G$ are continuous, $H$ is continuous too.
    We can deduce now that $H(\cdot, 0) = F(\cdot, 0) = f$ and $H(\cdot, 1) = G(\cdot, 1) = h$.
    Therefore, $f, h$ are homotopic. \\\\

    \noindent\textbf{Exercise 16.} \textit{Classify the letters of the alphabet into homotopy equivalence classes.} \\  

    \begin{itemize}
        \item The class of circles:
            \Huge
            $$\{\textrm{A}, \textrm{D}, \textrm{O}, \textrm{P}, \textrm{R}\}$$
            \normalsize

        \item The class of ``two holes'':
            \Huge
            $$\{\textrm{B}, \textrm{Q}\}$$
            \normalsize
        
        \item The class of points:
            \Huge
            \begin{gather*}
                \{\textrm{C}, \textrm{E}, \textrm{F}, \textrm{G}, \textrm{H}, \textrm{I}, \textrm{J}, \textrm{K}, \textrm{L}, \textrm{M},\\ \textrm{N}, \textrm{S}, \textrm{T}, \textrm{U}, \textrm{V}, \textrm{W}, \textrm{X}, \textrm{Y}, \textrm{Z}\}
            \end{gather*}
            \normalsize
    \end{itemize}

\end{document}
