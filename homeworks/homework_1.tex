\documentclass{article}
\usepackage[utf8]{inputenc}

\title{Homework 1\\
    \large{Topological Data Analysis with Persistent Homology}}
\author{Lucas Emanuel Resck Domingues\\    
    Professor: Raphaël Tinarrage\\\\
    {Escola de Matemática Aplicada}\\
    {Fundação Getulio Vargas}}
\date{\today}

% \usepackage{natbib}
\usepackage{graphicx}
\usepackage{amsmath}
\usepackage{amsfonts}

\begin{document}

    \maketitle

    \noindent\textbf{Exercise 4.} Consider a point $z \in \mathcal{B}\left(\dfrac{x+y}{2}, \dfrac{r}{2}\right)$, that is, $\left\lVert z - \dfrac{x+y}{2}\right\lVert < \dfrac{r}{2}$.
    We know that $\left\lVert x - \dfrac{x+y}{2}\right\lVert = \left\lVert\dfrac{x-y}{2} \right\lVert = \dfrac{r}{2}$. This way, we see that
    \begin{align*}
        \lVert z - x \lVert &= \left\lVert z - \dfrac{x+y}{2} + \dfrac{x+y}{2} - x \right\lVert\\
        &\le \left\lVert z - \dfrac{x+y}{2}\right\lVert + \left\lVert \dfrac{x+y}{2} - x \right\lVert\\
        &< \dfrac{r}{2} + \dfrac{r}{2}\\
        &= r
    \end{align*}
    Therefore, $z \in \mathcal{B}(x, r)$.
    
    Similarly, we can show that $z \in \mathcal{B}(y, r)$. So we conclude that $z \in \mathcal{B}(x, r) \cap \mathcal{B}(y, r)$, which means $\mathcal{B}\left(\dfrac{x+y}{2}, \dfrac{r}{2}\right) \subset \mathcal{B}(x, r) \cap \mathcal{B}(y, r)$.\\\\

    \noindent\textbf{Exercise 5.} Consider an open ball $\mathcal{B}(x, r)$ of $\mathbb{R}^n$.
    Take an arbitrary point inside it, let's call it $y$.
    So $\lVert x - y \lVert < r$.
    
    Now, consider the open ball $\mathcal{B}(y, d)$, with $d = r - \lVert x - y \lVert$.
    Because $\lVert x - y \lVert < r$, we have $d > 0$.
    If we take a point $z$ inside this open ball, we will have $\lVert y - z \lVert < d = r - \lVert x - y \lVert$.
    Well, it's the same of $\lVert y - z \lVert + \lVert x - y \lVert < r$, that is, by triangle inequality, $\lVert x - z \lVert < r$.
    We can say that $z \in \mathcal{B}(x, r)$ and $\mathcal{B}(y, d) \subset \mathcal{B}(x, r)$.
    
    We can conclude that, for any point $y$ of the open ball $\mathcal{B}(x, r)$, there is another open ball $\mathcal{B}(y, d)$ that contains $y$ but also $\mathcal{B}(y, d) \subset \mathcal{B}(x, r)$, that is, $\mathcal{B}(x, r)$ is open.

\end{document}
